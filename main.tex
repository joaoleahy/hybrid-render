\documentclass[aspectratio=169,xcolor=table]{beamer}
\usepackage{algorithm}
\usepackage{algpseudocode}
\usepackage[utf8]{inputenc}
\usepackage[T1]{fontenc}
\usepackage{lmodern}
\usepackage{csquotes}
\usepackage{xcolor}
\usepackage[portuguese]{babel}
\usepackage[backend=biber,style=numeric]{biblatex}
\addbibresource{Bibliografia.bib}

% ------------------------------------------------
% Tema do Beamer (exemplo de um tema customizado)
% ------------------------------------------------
\usetheme{DCC} % <-- Ajuste conforme seu tema ou estilo

\graphicspath{{imgs/}{../resultados/}}
\graphicspath{{imgs/}{../Figuras/}}

\author[Chaves, Leahy]{%
  \textbf{João Pedro Chaves} \\
  \textbf{João Victor Leahy}
}
\title{Renderização Híbrida: Combinando o Melhor de Dois Mundos \\ \large{MATA65 - Computação Gráfica}}
\institute{Universidade Federal da Bahia \\ Instituto de Computação}
\date{\today}

\begin{document}

%-------------------------------------------------
%  SLIDE DE TÍTULO
%-------------------------------------------------
\begin{frame}[plain,noframenumbering]
    \titlepage
\end{frame}

%-------------------------------------------------
%  SLIDE DE AGENDA
%-------------------------------------------------
\begin{frame}{Agenda}
    \tableofcontents
\end{frame}

%=================================================
\section{Introdução}
%=================================================
\begin{frame}{Introdução}
    \begin{itemize}
        \item \textbf{O que é renderização híbrida?}
        \begin{itemize}
            \item Combinação de técnicas de rasterização, computação e ray tracing para otimizar a qualidade visual e o desempenho em tempo real.
            \item Utiliza as capacidades únicas de cada etapa do pipeline gráfico.
        \end{itemize}
        \item \textbf{Por que usar a renderização híbrida?}
        \begin{itemize}
            \item Equilibra qualidade de imagem com performance em tempo real, aproximando a qualidade do path tracing offline.
            \item Permite modularidade e escalabilidade, adaptando-se às capacidades de hardware do usuário.
            \item Aumenta a qualidade visual ao usar ray tracing para efeitos que a rasterização não consegue gerar de forma precisa.
            \item Elimina a necessidade de soluções alternativas problemáticas (hacks) e algoritmos propensos a artefatos.
        \end{itemize}
    \end{itemize}
\end{frame}

%=================================================
\section{Técnicas e Etapas da Renderização Híbrida}
%=================================================
\begin{frame}{Visão Geral do Pipeline Híbrido}
    \begin{itemize}
        \item Renderização em espaço de objeto (object-space rendering)
        \item Ray tracing para transparência e translucidez
        \item Iluminação global (interreflexões difusas)
        \item G-buffer
        \item Sombras diretas (ray tracing ou rasterização)
        \item Reflexos (ray tracing ou computação)
        \item Iluminação direta (computação)
        \item Mesclagem de reflexos e radiosidade
        \item Pós-processamento
    \end{itemize}
\end{frame}

\begin{frame}{Sombras}
    \begin{itemize}
        \item Ray tracing para sombras precisas com tratamento de sombras duras e suaves.
        \item Utilização do G-buffer para reconstrução da posição da superfície no espaço do mundo.
        \item Suporte para sombras transparentes com ray tracing recursivo através de superfícies transparentes.
    \end{itemize}
\end{frame}

\begin{frame}{Reflexos}
    \begin{itemize}
        \item Ray tracing para reflexos dinâmicos e complexos.
        \item Lançamento de raios a partir do G-buffer, evitando ray tracing da visibilidade primária.
        \item Reconstrução e filtragem multiestágio para escalar reflexos para resolução total.
        \item Uso de amostragem estocástica e filtragem espaço-temporal para resultados mais precisos e uniformes.
    \end{itemize}
\end{frame}

\begin{frame}{Oclusão Ambiente e Transparência}
    \begin{itemize}
        \item \textbf{Oclusão Ambiente:}
        \begin{itemize}
            \item Ray tracing para oclusão ambiente de alta qualidade, livre das limitações das técnicas baseadas em rasterização.
            \item Amostragem estocástica da função de oclusão e filtragem para redução de ruído.
        \end{itemize}
        \item \textbf{Transparência e Translucidez:}
        \begin{itemize}
            \item Ray tracing para calcular o transporte de luz em meios espessos.
            \item Rastreamento de transições de interface e refrações realistas.
            \item Parametrização em espaço de textura para estabilidade e previsibilidade.
            \item Acumulação temporal para efeitos de translucidez.
        \end{itemize}
    \end{itemize}
\end{frame}

\begin{frame}{Iluminação Global}
    \begin{itemize}
        \item Soluções de iluminação difusa indireta sem pré-computação ou parametrização.
        \item Uso de surfels distribuídos dinamicamente no espaço do mundo para acumulação ao longo do tempo.
        \item Integração com oclusão ambiente de espaço de tela para compensar a falta de detalhes de alta frequência.
    \end{itemize}
\end{frame}

%=================================================
\section{Aplicações da Renderização Híbrida}
%=================================================
\begin{frame}{Aplicações}
    \begin{itemize}
        \item \textbf{Jogos:}
        \begin{itemize}
            \item Melhora a qualidade visual de sombras, reflexos, oclusão ambiente e outros efeitos.
            \item Permite o uso de ray tracing para efeitos de transparência e translucidez.
            \item Possibilita a renderização de iluminação global indireta em cenas dinâmicas.
            \item Aumenta a precisão de efeitos como profundidade de campo (DoF) com semitransparências mais realistas.
        \end{itemize}
        \item \textbf{Outras aplicações:}
        \begin{itemize}
            \item Visualização arquitetônica, design de produtos, realidade virtual e aumentada.
            \item Renderização de cenas dinâmicas com objetos e iluminação em mudança.
        \end{itemize}
    \end{itemize}
\end{frame}

%=================================================
\section{Otimização e Desafios}
%=================================================
\begin{frame}{Otimização e Desafios}
    \begin{itemize}
        \item \textbf{Importância da amostragem estocástica e filtragem:}
        \begin{itemize}
            \item Técnicas de acumulação temporal para melhorar a qualidade e estabilidade visual.
            \item Utilização de mapas de amostragem adaptativos para direcionar mais amostras a regiões de maior complexidade visual.
            \item Balanceamento entre qualidade e desempenho por meio do ajuste de orçamentos de raios e filtros.
        \end{itemize}
        \item \textbf{Desafios:}
        \begin{itemize}
            \item Gerenciamento de artefatos de ghosting devido à acumulação temporal.
            \item Problemas de tiling (repetição) devido a máscaras de raio insuficientes.
            \item Compensação para resultados enviesados de redes neurais em renderização estática.
            \item Necessidade de mais pesquisa sobre gerenciamento de nível de detalhe para jogos de mundo aberto.
        \end{itemize}
    \end{itemize}
\end{frame}

%=================================================
\section{Conclusão}
%=================================================
\begin{frame}{Conclusão}
    \begin{itemize}
        \item A renderização híbrida é uma técnica poderosa para alcançar alta qualidade visual em tempo real.
        \item Permite a combinação flexível de diferentes técnicas de renderização para obter os melhores resultados.
        \item Continua a evoluir com avanços em hardware e algoritmos, abrindo novas possibilidades para jogos e outras aplicações.
    \end{itemize}
\end{frame}

%=================================================
\section{Referências}
%=================================================
\begin{frame}{Referências}
    \begin{itemize}
        \item \fullcite{BarreBrisebois2019}
        \item \fullcite{Tan2020a}
        \item \fullcite{Kuznetsov2024}
    \end{itemize}
\end{frame}

\end{document}